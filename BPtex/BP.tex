\documentclass[a4paper,12pt,twoside]{report}
% nezbytné balíčky
\usepackage[T1]{fontenc}
\usepackage[utf8]{inputenc}  % vstupní znaková sada: UTF8
\usepackage[czech]{babel} % typografická pravidla     
\usepackage[a4paper, hmarginratio=3:2]{geometry} % využití celé A4 stránky a nastavení okrajů
\usepackage{graphicx} % balíček pro vkládání obrázků
\usepackage{tabularx} % rozšířené možnosti tabulek

% balíčky, které se mohou hodit
%\usepackage{amsmath} % balíček pro pokročilou matem. sazbu
%\usepackage{color} % pro možnost barevného textu
%\usepackage{fancybox} % umožňuje pokročilé rámečkování
%\usepackage{index} % nutno použít v případě tvorby rejstříku balíčkem makeindex
%\newindex{default}{idx}{ind}{Rejstřík} % zavádí rejstřík v případě použití balíku index
%\usepackage{caption} % pro popisky obrázků, tabulek atd.
%\usepackage{listings}  % balíček vhodný pro ukázky kódu 
%\usepackage{float} % rozšířené možnosti umístění obrázků
%\usepackage[hidelinks]{hyperref} % pro klikací odkazy v pdf
%\usepackage{encxvlna} % postará se o spojky a předložky, které dle českých pravidel nesmí být na konci řádku. Dokumentace: http://texdoc.net/texmf-dist/doc/generic/encxvlna/encxvlna.pdf

\topmargin=-15mm      % horní okraj trochu menší
\textwidth=150mm      % šířka textu na stránce
\textheight=240mm     % "výška" textu na stránce

\pagenumbering{arabic} % číslování stránek arabskými číslicemi
\pagestyle{plain}      % stránky číslované dole uprostřed

\parindent=0pt % odsazení 1. řádku odstavce
\parskip=7pt   % mezera mezi odstavci
\frenchspacing % aktivuje použití některých českých typografických pravidel

\newcommand{\ti}{\textit} %zkrácený příkaz pro kurzívu
\newcommand{\tb}{\textbf} %zkrácený příkaz pro tučné


%%%%%%%%%%%%%%%%%%%%%% zde jsou zavedeny některé "konstanty" - můžete, resp. musíte je ZMĚNIT %%%%%%%%%%%%%%%%%%%%%%
\newcommand{\cvut}{České vysoké učení technické v Praze}
\newcommand{\fjfi}{Fakulta jaderná a fyzikálně inženýrská}
\newcommand{\kse}{Katedra softwarového inženýrství v ekonomii}
\newcommand{\obor}{Inženýrská informatika}
\newcommand{\zamereni}{Softwarové inženýrství v ekonomii}

\newcommand{\nazevcz}{Modely zátěže výpočetních serverů}        % zde VYPLŇTE český název práce (přesně podle zadání!)
\newcommand{\nazeven}{Computational requirements modelling}     % zde VYPLŇTE anglický název práce (přesně podle zadání!)
\newcommand{\autor}{Dmitriy Burdin}           % zde VYPLŇTE své jméno a příjmení
\newcommand{\rok}{2014}          % zde VYPLŇTE měsíc a rok odevzdání, např. May 2011
\newcommand{\vedouci}{Ing. Jan Doubek}         % zde VYPLŇTE jméno a příjmení vedoucího práce, včetně titulů, např. Doc. Ing. Ivo Malý, Ph.D.

\newcommand{\druh}{BAKALÁŘSKÁ PRÁCE}

\newcommand{\pracovisteVed}{CISCO Systems s.r.o.}			% zde VYPLŇTE pracoviště vedoucího práce

\newcommand{\konzultant}{-} % POKUD MÁTE určeného konzultanta, NAPIŠTE jeho jméno a příjmení
\newcommand{\pracovisteKonz}{-} % POKUD MÁTE konzultanta, NAPIŠTE jeho pracoviště


\newcommand{\klicova}{Časová řada, ekonometrie, předpovídání, analýza, server}   % zde NAPIŠTE česky max. 5 klíčových slov
\newcommand{\keyword}{Times series, econometrics, forecasting, analysis, server}       % zde NAPIŠTE anglicky max. 5 klíčových slov (přeložte z češtiny)
\newcommand{\abstrCZ}{Cílem této práce je efektivně modelovat časovou, výpočetní a paměťovou náročnost distribuovaných úloh. Implementace modelů na základě časových řad výsledků minulých úloh. Výsledkem práce bude studie použitelnosti ekonometrických metod v prostředí velkých data center. Dále návrh algoritmů pro implementaci studovaných metod. Výstupní modely budou použity pro lepší plánování rozvržení výpočetních zdrojů.} % zde NAPIŠTE abstrakt v češtině
\newcommand{\abstrEN}{The goal of this thesis is to effectively make a model of temporal, computational and memory requirements of distributed tasks. Model implementation based on time series results of past tasks. The result of the thesis will be a study of applicability of econometric methods in large scale data centres. In addition, design of algorithms for implementation of the studied methods. Output models will be used for better planning of computational sources arrangement.}                  % zde NAPIŠTE abstrakt v angličtině

\begin{document}

%%%%%%%%%%%%%%%%%%%%%% Titulní strana -- na následujících 30 řádků NESAHEJTE!!!  Generuje se AUTOMATICKY %%%%%%%%%%%%%%%%%%%%%%
\thispagestyle{empty}

\begin{center}
    {\Large \textsc{\cvut}\\[1.5ex] \textsc{\fjfi}}\\
    \vspace{10mm}

    \begin{tabular}{c}
	    {\bf \kse}\\   
      {\bf Obor: \obor}\\
      {\bf Zaměření: \zamereni}\\
    \end{tabular}

   % logo CVUT -- pokud jej nechcete použít, zakomentujte následující řádek a odkomentujte řádek pod ním:
   \vspace{10mm} \includegraphics[height=25mm]{lev.pdf} \vspace{15mm}
   % \vspace{50mm}

   {\huge \bf \nazevcz}\\
   \vspace{5mm}   
   {\huge \bf \nazeven}
   
   \vspace{15mm}
   {\Large \druh}

   \vfill
   {\large
    \begin{tabular}{ll}
    Vypracoval: & \autor\\
    Vedoucí práce: & \vedouci\\
    Rok: & \rok
    \end{tabular}
   }
\end{center}

%%%%%%%%%%%%%%%%%%%%%% Zadání práce  %%%%%%%%%%%%%%%%%%%%%%
%%%%%%%%%%%%%%%%%%%%%%            Před svázáním namísto této strany VLOŽÍTE zadání podepsané děkanem!
\newpage  % SEM NESAHEJTE!
\thispagestyle{empty} % SEM NESAHEJTE!

Před svázáním místo téhle stránky \fbox{vložíte zadání práce} s podpisem
děkana a do pdf verze oskenované zadání.

%oskenované zadání lze vložit jako obrázek

% 1. strana zadání
%\begin{center}
%     \includegraphics[width=1\textwidth]{zadani1.jpg}
%\end{center}

% 2. strana zadání
%\newpage  % SEM NESAHEJTE!
%\thispagestyle{empty} % SEM NESAHEJTE!
%\begin{center}
%     \includegraphics[width=1\textwidth]{zadani2.jpg}
%\end{center}

%%%%%%%%%%%%%%%%%%%%%% Prohlášení -- ŽENY UPRAVÍ minulý čas sloves %%%%%%%%%%%%%%%%%%%%%%
\newpage % SEM NESAHEJTE!
\thispagestyle{empty}  % SEM NESAHEJTE!

~ % SEM NESAHEJTE!
\vfill % prázdné místo. SEM NESAHEJTE!

{\bf Prohlášení} % SEM NESAHEJTE!

\vspace{0.5cm} % vertikální mezera. SEM NESAHEJTE!
Prohlašuji, že jsem svou bakalářskou práci vypracoval samostatně a použil jsem pouze podklady
(literaturu, projekty, SW atd.) uvedené v přiloženém seznamu.

\vspace{5mm}  % SEM NESAHEJTE!
\begin{tabularx}{\textwidth}{X c}                               	% SEM NESAHEJTE!
    V Praze dne .................... &........................................ \\	% SEM NESAHEJTE!
	& \autor
\end{tabularx}	% SEM NESAHEJTE!

%%%%%%%%%%%%%%%%%%%%%% Poděkování -- UPRAVTE JMÉNO, resp. tuto stránku celou VYMAŽTE %%%%%%%%%%%%%%%%%%%%%%
%%%%%%%%%%%%%%%%%%%%%%                           (poděkování nemusí být uvedeno vůbec)
\newpage
\thispagestyle{empty}

~
\vfill % prázdné místo

{\bf Poděkování}

\vspace{5mm} % vertikální mezera
Děkuji Ing. Janu Doubkovi, Ph.D. za vedení mé bakalářské práce a za podnětné návrhy, které ji obohatily.

\begin{flushright}
\autor
\end{flushright}  % <------- tady končí stránka s poděkováním

%%%%%%%%%%%%%%%%%%%%%% Abstrakt atp. Je generován AUTOMATICKY podle údajů na začátku souboru) %%%%%%%%%%%%%%%%%
\newpage   % SEM NESAHEJTE!
\thispagestyle{empty}   % SEM NESAHEJTE!

% příprava:    (na následujících 8 řádků NESAHEJTE!)
\newbox\odstavecbox
\newlength\vyskaodstavce
\newcommand\odstavec[2]{%
    \setbox\odstavecbox=\hbox{%
         \parbox[t]{#1}{#2\vrule width 0pt depth 4pt}}%
    \global\vyskaodstavce=\dp\odstavecbox
    \box\odstavecbox}
\newcommand{\delka}{120mm} % šířka textů ve 2. sloupci tabulky

% použití přípravy:    % dovnitř "tabular" vůbec NESAHEJTE!
\begin{tabular}{ll}
  {\em Název práce:} & ~ \\
  \multicolumn{2}{l}{\odstavec{\textwidth}{\bf \nazevcz}} \\[0.5em]
  {\em Autor:} & \autor \\[0.5em]
  {\em Obor:} & \obor \\[0.5em]
  {\em Druh práce:} & \druh \\[0.5em]
  {\em Vedoucí práce:} & \odstavec{\delka}{\vedouci \\ \pracovisteVed} \\[0.5em]
  %{\em Konzultant:} & \odstavec{\delka}{\konzultant \\ \pracovisteKonz} \\[0.5em] % ZAKOMENTUJTE v případě, že jste neměli konzultanta
 \\[0.01em]  
  \multicolumn{2}{l}{\odstavec{\textwidth}{{\em Abstrakt:} ~ \abstrCZ  }} \\[0.5em]
 \\[0.01em]
  {\em Klíčová slova:} & \odstavec{\delka}{\klicova} \\[2em]

  {\em Title:} & ~\\
  \multicolumn{2}{l}{\odstavec{\textwidth}{\bf \nazeven}}\\[0.5em]
  {\em Author:} & \autor \\[0.5em]
  \multicolumn{2}{l}{\odstavec{\textwidth}{{\em Abstract:} ~ \abstrEN  }} \\[0.5em]
 \\[0.1em]
  {\em Key words:} & \odstavec{\delka}{\keyword}
\end{tabular}



%%%%%%%%%%%%%%%%%%%%%% Obsah práce ... je generován AUTOMATICKY %%%%%%%%%%%%%%%%%%%%%%
\newpage  % SEM NESAHEJTE!
\tableofcontents % SEM NESAHEJTE!


%%%%%%%%%%%%%%%%%%%%%%  Zde začíná SAMOTNÁ PRÁCE  %%%%%%%%%%%%%%%%%%%%%%%%%%%%%%%%%%%%%%%%%%%%
\newpage % SEM NESAHEJTE!

\chapter*{Úvod}
\addcontentsline{toc}{chapter}{Úvod} % SEM NESAHEJTE!
%
%  Sem napiště úvod
%

\chapter{Název kapitoly}
%
% Tady začněte psát vlastní práci
%

\chapter*{Závěr}
\addcontentsline{toc}{chapter}{Závěr} % SEM NESAHEJTE!
%
% Sem napiště závěr
%


\clearpage
\addcontentsline{toc}{chapter}{Literatura} % SEM NESAHEJTE!
\begin{thebibliography}{99}   
	\bibitem FFumio Hayashi. \emph{Econometrics}. Princeton. Princeton University Press. 2000.
\end{thebibliography}

\newpage % SEM NESAHEJTE!
\addcontentsline{toc}{chapter}{Přílohy} % SEM NESAHEJTE!
\appendix % SEM NESAHEJTE!

\chapter{Název přílohy} % SEM NESAHEJTE!
%
% Zde uveďte seznam příloh
%


\end{document} % SEM NESAHEJTE!
